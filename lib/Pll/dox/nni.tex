% % This is LLNCS.DEM the demonstration file of
% the LaTeX macro package from Springer-Verlag
% for Lecture Notes in Computer Science,
% version 2.4 for LaTeX2e as of 16. April 2010
%
\documentclass[oneside,12pt]{book}
%\usepackage[top=1cm,bottom=3cm,height=14cm,paperheight=14cm,width=18cm,paperwidth=18cm]{geometry}
\usepackage[paperheight=16cm,paperwidth=23cm]{geometry}
\usepackage[absolute]{textpos}
%
\usepackage[pst,vaucanson-g]{xcolor}
% vlined
\usepackage{pstricks}
\usepackage{auto-pst-pdf,pstricks-add}
%\usepackage{enumerate}
\usepackage{pstricks}
\usepackage{pst-tree}
%\usepackage{subfigure}
\usepackage{graphicx}
%\usepackage{subfig}
%\usepackage{tabularx}
%\usepackage{pstricks-add}
\usepackage{amssymb}
\usepackage{amsmath}
\usepackage{vaucanson-g}
%\usepackage{pstcol}
\usepackage{url}
\usepackage{comment}
\usepackage{a4wide}
%\usepackage{tikz}
%\usetikzlibrary{trees}
%\usetikzlibrary{decorations.pathmorphing}
%\usetikzlibrary{decorations.markings}
%\usetikzlibrary{arrows}
%\usepackage{tikz}
%\newcommand*\circled[1]{\tikz[baseline=(char.base)]{
%  \node[shape=circle,draw,inner sep=0.5pt] (char) {#1};}}

% \makeatletter
% \def\zaptype#1{%
% \listsubcaptions % Finish the last set of sub-floats before
% \def\@captype{#1}}% switching to another float type.
% \makeatother

\thispagestyle{empty}
\begin{document}
\begin{textblock}{3}(0,0.5)
\scalebox{1.5}{
\begin{pspicture}[showgrid=false](15,11)
\psStartPoint[A](6,6) % nodes have the base name A
\psVector[arrows=-](2;0)\psVector[arrows=-](1.5;0)\psVector[arrows=-](0.5;0)\psVector[arrows=-](0.5;60)\psVector[arrows=-,linestyle=dashed,linecolor=red](2.5;60)

\psStartPoint[tmp](10,6)
\psVector[arrows=-](0.5;60)\psVector[arrows=-,linestyle=dashed,linecolor=red](2;60)\psVector[arrows=-](0.5;60)
\psVector[arrows=-](1.5;0)\psVector[arrows=-,linestyle=dashed](1;0)
\psStartPoint[tmp](10,6)
\psVector[arrows=-](0.5;60)\psVector[arrows=-,linestyle=dashed, linecolor=red](2;60)\psVector[arrows=-](0.5;60)
\psVector[arrows=-](1.5;120)\psVector[arrows=-,linestyle=dashed](1;120)

\psline[origin=A5](0.5;0)
\psline[origin=A5](0.5;120)
\psarc[linestyle=dashed]{->}(A5){0.5}{0}{120}
\psarc[linestyle=dashed]{->}(A5){0.5}{120}{240}
\psarc[linestyle=dashed]{->}(A5){0.5}{240}{0}
\psdot[origin=A5](0.5;0)
\psdot[origin=A5](0.5;120)
\psdot[origin=A5](0.5;240)

%\psStartPoint[B](6,6)
\pnode(A3){T1}
\psStartPoint[B](10,6)
\psVector[arrows=-](0.5;-60)\psVector[arrows=-](2;-60)\psVector[arrows=-](0.5;-60)\psVector[arrows=-](0.5;-120)
\psdot(A2)
\psdot(A4)
\psarc[linestyle=dashed]{<-}(A3){0.5}{60}{180}
\psarc[linestyle=dashed]{<-}(A3){0.5}{180}{300}
\psarc[linestyle=dashed]{<-}(A3){0.5}{300}{60}


\uput[135](A2){$q$}
\psarc[linestyle=dashed]{->}(B3){0.5}{120}{240}
\psarc[linestyle=dashed]{->}(B3){0.5}{240}{0}
\psarc[linestyle=dashed]{->}(B3){0.5}{0}{120}

\psdot[origin=B3](0.5;0)
\psdot[origin=B3](0.5;240)
\psline[origin=B3](0.5;0)

%extensions of r 
\psStartPoint[C](10,6)
\psVector[arrows=-](0.5;-60)\psVector[arrows=-](2;-60)\psVector[arrows=-](0.5;-60)
\psVector[arrows=-](1.5;0)\psVector[arrows=-,linestyle=dashed](1;0)
\psStartPoint[D](10,6)
\psVector[arrows=-](0.5;-60)\psVector[arrows=-](2;-60)\psVector[arrows=-](0.5;-60)
\psVector[arrows=-](1.5;-120)\psVector[arrows=-,linestyle=dashed](1;-120)

%%%%%%%%%%%%%% DRAW THE LEFT PART OF THE TREE

%upper part
\psStartPoint[E](6,6)
\psVector[arrows=-](1.5;180)\psVector[arrows=-](0.5;180)\psVector[arrows=-](0.5;120)\psVector[arrows=-,linecolor=red,linestyle=dashed](2;120)\psVector[arrows=-](0.5;120)\psVector[arrows=-](0.5;180)\psVector[arrows=-](1;180)\psVector[arrows=-,linestyle=dashed](1;180)
\psStartPoint[F](6,6)
\psVector[arrows=-](1.5;180)\psVector[arrows=-](0.5;180)\psVector[arrows=-](0.5;120)\psVector[arrows=-,linecolor=red,linestyle=dashed](2;120)\psVector[arrows=-](0.5;120)\psVector[arrows=-](0.5;60)\psVector[arrows=-](1;60)\psVector[arrows=-,linestyle=dashed](1;60)
\uput[270](F4){$r$}
\psdot(F6)
\psarc[linestyle=dashed]{->}(E5){0.5}{60}{180}
\psarc[linestyle=dashed]{->}(E5){0.5}{180}{300}
\psarc[linestyle=dashed]{->}(E5){0.5}{300}{60}

\uput[45](E1){$p$}
\psStartPoint[G](4,6)
\psVector[arrows=-](0.5;-120)\psVector[arrows=-,linestyle=dashed,linecolor=red](2;-120)\psVector[arrows=-](0.5;-120)\psVector[arrows=-](0.5;-180)\psVector[arrows=-](1;-180)\psVector[arrows=-,linestyle=dashed](1;-180)
\uput[90](G2){$s$}
\psStartPoint[H](4,6)
\psVector[arrows=-](0.5;-120)\psVector[arrows=-,linecolor=red,linestyle=dashed](2;-120)\psVector[arrows=-](0.5;-120)\psVector[arrows=-](0.5;-60)\psVector[arrows=-](1;-60)\psVector[arrows=-,linestyle=dashed](1;-60)
\psarc[linestyle=dashed]{->}(G3){0.5}{60}{180}
\psarc[linestyle=dashed]{->}(G3){0.5}{180}{300}
\psarc[linestyle=dashed]{->}(G3){0.5}{300}{60}

\psarc[linestyle=dashed]{->}(E2){0.5}{0}{120}
\psarc[linestyle=dashed]{->}(E2){0.5}{120}{240}
\psarc[linestyle=dashed]{->}(E2){0.5}{240}{0}

\pscurve[showpoints=false,linecolor=blue,linestyle=dashed]{-}(F3)(9,7)(tmp2)
\pscurve[showpoints=false,linecolor=green,linestyle=dashed]{-}(G1)(11,5)(tmp2)

\pscurve[showpoints=false,linecolor=blue,linestyle=dashed]{-}(F4)(9,8)(tmp1)
\pscurve[showpoints=false,linecolor=green,linestyle=dashed]{-}(G2)(11,4)(tmp1)
\psdot(B1)
\psdot(B2)
\psdot(G1)
\psdot(G2)
\psdot(G4)
\psdot(H4)
\psdot(E1)
\psdot(E3)
\psdot(E4)
\psdot(E6)
\psdot(tmp1)
\psdot(tmp2)



% Do the traversals now
% Upper left tree
%\psset{linestyle=none}
\psStartPoint[TIP](4,6)
\psVector[arrows=-,linestyle=solid](140;4)

%%%%%%%%%%%% SECOND TREE %%%%%%%%%%%%%%%%%%

%%%%% bottom part %%%%%
%\psStartPoint[D](12,6)
%\psVector[arrows=-](0.5;270)\psVector[arrows=-,linestyle=dashed,linecolor=red](2;270)\psVector[arrows=-](0.5;270)\psVector[arrows=-](1.5;210)\psVector[linestyle=dashed,arrows=-](1;210)
%\psStartPoint[E](12,3)
%\psVector[arrows=-](1.5;-30)\psVector[linestyle=dashed,arrows=-](1;-30)
%\psarc[linestyle=dashed]{->}(D3){0.5}{90}{210}
%\psarc[linestyle=dashed]{->}(D3){0.5}{210}{330}
%\psarc[linestyle=dashed]{->}(D3){0.5}{330}{90}
%\psdot[origin=D3](0.5;90)
%\psdot[origin=D3](0.5;210)
%\psdot[origin=D3](0.5;330)
%\uput[45](D2){$q$}
%
%%%%% upper part %%%%%
%\psStartPoint[F](12,6)
%\psVector[arrows=-](1.5;150)\psVector[linestyle=dashed,arrows=-](1;150)
%\psStartPoint[F](12,6)
%\psVector[arrows=-](1.5;30)\psVector[linestyle=dashed,arrows=-](1;30)
%\psarc[linestyle=dashed]{->}(D0){0.5}{150}{270}
%\psarc[linestyle=dashed]{->}(D0){0.5}{270}{30}
%\psarc[linestyle=dashed]{->}(D0){0.5}{30}{150}
%\psdot[origin=D0](0.5;150)
%\psdot[origin=D0](0.5;270)
%\psdot[origin=D0](0.5;30)
%\uput[315](D1){$r$}
%
%%\cnode[origin=B3](0.5;0){1pt}{P1}
%\psset{origin=B3}
%%\psset{origin={0,0}}
%\pscurve[showpoints=false,linecolor=blue,linestyle=dashed]{-}(0.5;240)(1.5,-1)(4.5,0.5)(D2)
%\pscurve[showpoints=false,linecolor=blue,linestyle=dashed]{-}(0.5;0)(3.5,1)(D1)
%\psset{origin={0,0}}
\end{pspicture}}
\end{textblock}
\end{document}
